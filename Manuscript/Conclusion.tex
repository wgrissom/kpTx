%!TEX root = kPtx_paper.tex
\section*{Discussion and Conclusions}
In practice, error near the transition band can be mitigated by ensuring the region of interest (ROI) of the cortex falls completely in the stop-band and assuming zero signal within the pass-band during undersampled reconstruction. 
A computationally efficient and highly parallelizable k-space domain pTx pulse design method was proposed.
The method adapts the k-space domain formulism proposed by Katscher et al \cite{katscher2003transmit}, which does not require the computationally expensive iterative processes in the conventional spatial domain design methods. The method advances the work of Katscher et al \cite{katscher2003transmit} by finely parallelizing the process of finding the system matrix. Furthermore, a technique inspired by a rapid GRAPPA calibration method \cite{luo2019grappa} is also included to further speed up the computation of each instance. The proposed k-space domain design method was evaluated through simulation upon an inner volume pulse design problem for MR Corticography, which can in turn enable submillimeter whole-cortex and cortex-specific imaging. The optimal parallelization parameters: thread number, patch width, and inclusion width, were found for this design problem. The proposed method reduced the computation time by 97\% compared to the conventional spatial domain design method, while producing equal inner volume suppression performance. And the abilities of the proposed method to accommodate k-space undersampling and correct off-resonance were also demonstrated to be equal to conventional spatial domain design method.