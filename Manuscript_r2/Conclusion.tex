%!TEX root = kPtx_paper.tex
\section*{Discussion}
\subsection*{Summary}
A small-tip-angle k-space domain parallel transmit pulse design algorithm was proposed
that divides up the calculation of a matrix relating a target excitation pattern to the pulses that produce it 
into a set of independent problems for patches of target excitation k-space locations,
each of which is influenced by a local neighborhood of points on the excitation k-space trajectory.
The division of the problem into patches of target locations creates an opportunity for fine parallelization,
while the limited neighborhood sizes lead to small problem sizes for each patch. 
Compared to the original k-space-based algorithm of Ref. \cite{Katscher:2003:Magn-Reson-Med:12509830},
the L-curve and matrix size results showed that 
the new algorithm produces much smaller matrix sizes that can be calculated more quickly,
with the tradeoff of increased excitation error or RF power. 
Results showed that the algorithm also enables compensation of off-resonance 
which has not previously been described in a k-space domain design. 
Compared to a spatial domain design,
the new algorithm is non-iterative and can be finely parallelized to achieve shorter design times, 
and results showed that it can use coarser target grid sizes while avoiding Gibbs ringing, 
again with the tradeoff of increased excitation error or RF power.
The performance of off-resonance-compensated spatial domain and k-space domain pulse designs was similar,
and the methods were similarly sensitive to excitation k-space undersampling. 
While all the pulse designs in this work used 3D SPINS trajectories,
the method can be applied in any number of dimensions and with any excitation k-space trajectory.
The MATLAB implementation described can be used for any two- or three-dimensional pulse design without modification.

\subsection*{Applications and Extensions}
This work was initially motivated by the observation that spatial domain parallel pulse designs can be very slow
for 3D problems with large grid sizes, 
requiring many iterations with considerable computation per iteration. 
It is anticipated that the proposed k-space domain algorithm will be most useful for these types of large $>$2D problems,
which include 3D spatial designs \cite{malik2012tailored,davids2016fast} 
and 2D and 3D spatial-spectral designs \cite{stenger2000three,setsompop2009,Malik:2010aa,yang2010four}
where full matrix construction and inversion is infeasible due to the problem size, 
and an iterative design can require several minutes to solve. 
Furthermore, unlike an iterative spatial domain design the proposed algorithm does not need to be repeated if the target pattern changes.
This means it could have a considerable computational speed advantage for Gerchberg-Saxton magnitude least-squares pulse designs \cite{setsompop2008magnitude,malik:mrm:2015}
which alternate between designing pulses and updating a complex-valued target pattern.
Such a design method would allow the user to specify only the magnitude of the target pattern, 
rather than magnitude and phase as was required in this work.
The method could also be used to initialize spatial-domain designs to reduce the number of iterations required to reach a target cost. 
Finally, while simple Tikhonov RF power regularization was used in the designs presented here,
more sophisticated regularization could be incorporated to, 
e.g., control peak RF power via adaptive regularization \cite{Yip:2005:Magn-Reson-Med:16155881},
or to enforce array compression by projecting the weights into the null space of a compression matrix \cite{cao2016array},
among other applications \cite{padormo:2016,deniz:2019}.
In such designs, it would be beneficial to pre-compute and store the lower triangular elements of the $\bm{S}^H\bm{S}$ matrices
so they need not be re-computed as the regularization changes over iterations.  
Peak power could also be controlled in k-space domain designs using parallel transmission 
VERSE \cite{Lee:2011:MRM} or the iterative re-VERSEit technique \cite{lee2009tod}.
It is not yet clear whether or how hard constraints on SAR and 
peak- and integrated-power could be directly incorporated in the k-space method, as in Refs \cite{brunner2010optimal} and \cite{hoyos:tmi:2014}.
It may also be possible to use the k-space domain method to rapidly design large-tip-angle pulses via the direct linear class of large-tip-angle pulses method \cite{Xu:2008aa} or the additive angle method \cite{grissom:mrm:2008}; 
the latter alternates between small-tip-angle designs that could be solved by the k-space method, 
and Bloch simulations to update the target pattern.


\section*{Conclusion}
The proposed k-space domain algorithm accelerates and finely parallelizes parallel transmission pulse design,
with a modest tradeoff of excitation error and RMS RF amplitude.


\section*{Acknowledgments}
The authors would like to thank Tianrui Luo (University of Michigan) for helpful discussions regarding Ref \cite{luo2019grappa}.