%!TEX root = kPtx_paper.tex
\section*{Results}


Figure \ref{fig:ErrorMap} shows simulated excitation patterns and error maps for the reference spatial and k-space domain designs. 
Both designs were performed with 16 parallel threads and the 64$\times$64$\times$48 grid size (3 mm isotropic resolution), 
and the excitation patterns were evaluated against the target pattern on a 128$\times$128$\times$96 grid size (1.5 mm iso-resolution). 
The k-space domain design used patch and inclusion widths of 4.
The calculated root-mean-squared errors (RMSEs) were 2.13\% (spatial domain) and 2.31\% (k-space domain), 
respectively. 
For both design methods, most of the errors appeared at the edges of the transition band, 
and errors elsewhere were lower than 5\% of the target flip angle. 
This indicates uniform inner volume excitation while maintaining the outer volume intact. 
The parallelized k-space domain design required 6.7 seconds computation versus 30.3 seconds for the spatial domain method, a 78\% decrease.


\subsection*{k-Space Algorithm Parameters}
Figure \ref{fig:ComputationTime}a plots mean computation time versus number of parallel threads,
holding the patch and inclusion widths fixed at 4 cycles/FOV. 
The computation time decreased rapidly with increasing thread number up to 12 threads, 
and then plateaued, likely due to the overhead involved in initiating threads after that point.
Based on this result, the number of threads was held fixed at 16 for subsequent designs when off-resonance was not compensated.

\par Figure \ref{fig:ComputationTime}b plots mean computation time and RMSE with different patch widths. 
The computation time decreased up to a patch width of 4 cycles/FOV (corresponding to $4^3 = 64$ simultaneously solved columns of $\bm{W}$) 
and then increased sharply for larger patch widths. 
RMSE decreased slowly as the patch width increased since the number of excitation trajectory points included in the calculation of weights for each 
target point increases on average (and especially for target locations in the middle of each patch) as the patch width increases, 
even when the inclusion width stays fixed.
Based on this result, a patch width of 4 cycles/FOV was used for subsequent designs.

\par Figure \ref{fig:ComputationTime}c plots mean computation time and RMSE with different inclusion widths. 
Computation time increased and error decreased with increasing inclusion width,
since more excitation trajectory points were included in each target location's calculation for increasing inclusion width,
corresponding to an increased $\bm{S}^H\bm{S}$ matrix size. 
The knees in the curves occurred approximately at an inclusion width of 4 cycles/FOV,
so this value was used in subsequent designs. 

\par Table \ref{fig:wsize} lists the size of the final matrix $\bm{W}$ in gigabytes,
versus inclusion width.
As inclusion width increases, more excitation trajectory points are used in the solution of the weights for each target location,
until the entire trajectory is used for each location (inclusion width = $\infty$ in the table),
corresponding to a full solution. 
With the inclusion width of 4 cycles/FOV used here, 
the matrix size was 99\% smaller than that of a full solution.  





\subsection*{L-Curves}
Figure \ref{fig:LCurves} plots flip angle RMSE versus RF RMS amplitude for the spatial domain method
and different configurations of the k-space domain method.
The spatial domain method achieves the best overall tradeoff between error and RMS amplitude (dashed black curve),
which is matched by the k-space domain method when all target locations are solved simultaneously with
all excitation trajectory points included and exact $\bm{S}^H\bm{S}$ matrix construction (solid orange curve). 
When the inclusion and patch widths are limited to 4 cycles/FOV but the $\bm{S}^H\bm{S}$ 
matrices are still constructed exactly, there is an increase in error and RMS amplitude (solid blue curve). 
However, a larger penalty is incurred by interpolating the entries of the $\bm{S}^H\bm{S}$ matrices (dashed orange curve)
than by limiting the inclusion and patch widths.
Combining interpolation of the $\bm{S}^H\bm{S}$ matrix entries and patch and inclusion widths of 4 cycles/FOV yield 
the dashed blue curve, which has a higher error and RF amplitude than when interpolation or small patch and inclusion widths are
used alone.
Overall, these results and the results in Figure \ref{fig:ComputationTime} and Table \ref{fig:wsize}
show that the k-space domain method allows a tradeoff between computation time and memory usage versus
excitation error and RMS RF amplitude.
The dots on the spatial domain and k-space domain curves indicate the knees \textcolor{blue}{of}\revbox{R1.9} the curves corresponding to
the $\lambda$ values used for the designs in Figures \ref{fig:ErrorMap}, \ref{fig:ComputationTime}, \ref{fig:kspace_PTX_Acceleration}, and \ref{fig:kspace_PTX_B0}.
Note that the RMSE's in Figure \ref{fig:LCurves} are slightly lower for the spatial domain designs
and slightly higher for the k-space domain designs compared to other figures,
because the flip angle errors in Figure \ref{fig:LCurves} were calculated using 
spatial domain non-uniform discrete Fourier transforms instead of Bloch equation simulations, 
to provide a more direct measure of the design error. 



\subsection*{Gibbs Ringing}
Figure \ref{fig:GibbsRing}a shows slices of the excitation error pattern produced by pulses designed by the spatial domain method
on a 32$\times$32$\times$24 grid, which were Bloch-simulated on the original 128$\times$128$\times$96 grid. 
There is significant Gibbs ringing in the pattern (indicated by the red arrows),
and the pulses incur a higher RMSE (4.90\%) than pulses designed using either the spatial domain method with a finer 64$\times$64$\times$48 grid
(2.43\%; Figure \ref{fig:GibbsRing}b) or the k-space domain method with a 32$\times$32$\times$24 grid (2.96\%; Figure \ref{fig:GibbsRing}c). 
Gibbs ringing is not apparent in either the 64$\times$64$\times$48 spatial domain error pattern or the 32$\times$32$\times$24 k-space domain error pattern. 
The RMS RF amplitudes of the low resolution designs were both 0.004. 
Note that the error of the 64$\times$64$\times$48 spatial domain design is slightly higher than designs presented in other figures
due to the lower-resolution excitation trajectory. 
From a spatial domain point of view, 
the Gibbs ringing in the low-resolution spatial domain design was caused by the design's inability to observe and limit the ringing in the low-resolution design grid. 
From a k-space domain point of view, the Gibbs ringing was due to implicit circulant end conditions at the edges of excitation k-space,
which led RF samples at one edge of k-space to wrap-around and affect target locations at the opposite edge of k-space. 
This effect is mitigated using a high-resolution spatial domain design grid, as illustrated in Figure \ref{fig:GibbsRing}b. 
However, even for a low-resolution k-space domain design there is no wrap-around effect in excitation k-space 
because the trajectory points that are incorporated in solution for each patch are explicitly specified to be those in the immediate vicinity of 
that patch, without circulant end conditions. 
%Particularly, for the instance problems whose patches are at the edges of the k-space FOV, the excitation trajectory points at the other ends of the k-space FOV will not be incorporated into the design.    

%Due to implicit circulant end conditions in k-space for the spatial domain design, t



\subsection*{Excitation k-Space Undersampling}
Figure \ref{fig:kspace_PTX_Acceleration} compares excitation error patterns produced by spatial domain-designed pulses (top row)
and k-space domain-designed pulses (middle row),
for different trajectory reduction factors which resulted in the SPINS trajectories plotted in the third row. 
For each design the k-space domain-designed pulses had higher error, 
but error increased smoothly with increasing reduction factor, 
as it did for the spatial domain designs. 
\textcolor{blue}{Figure \ref{fig:kspace_PTX_Acceleration} also reports mean computation time for each design.
The spatial domain times were fairly constant across reduction factors despite the decreasing number of time samples with increasing reduction factor.
This was because the size of the FFT operations in the NUFFTs 
depended not on the length of the trajectory but on the size of the target pattern which did not change with reduction factor,
and the FFT operations dominated the NUFFT computation times relative to the gridding operations. % for these three-dimensional problems,
Conversely, the k-space domain times did decrease with reduction factor since the number and sizes of the $\bm{S}^H\bm{S}$
matrices depend on the trajectory length.} \revbox{R2.1}



\subsection*{Off-Resonance}
Figure \ref{fig:kspace_PTX_B0}a shows the off-resonance maps containing a Gaussian distortion which was centered above the frontal sinus,
and scaled to peak amplitudes of 0, 200, and 400 Hz for the pulse designs and Bloch simulations. 
Figure \ref{fig:kspace_PTX_B0}b shows excitation error patterns and RMSEs for spatial domain designs with off-resonance compensation,
and k-space domain designs without and with off-resonance compensation. 
Without off-resonance compensation, 
the k-space domain-designed pulses produced large ($>$ 10\% of $M_0$) excitation errors both inside and outside the target ellipse. 
When the off-resonance map was incorporated in the spatial domain and k-space domain designs, 
the distortion was nearly fully corrected when it had a peak amplitude of 200 Hz. 
When the map was scaled to a peak of 400 Hz, some large errors remained, with the k-space domain-designed pulses achieving slightly lower RMSE. 
%It was also able to provide excitation pattern comparable to spatial domain design with 400 Hz maximum off-resonance. The k-space domain design has a weaker performance with higher off-resonance due to the fact that the time segmentation approximation in Ref \cite{fessler2005toeplitz} was meant to approximate forward models from RF to excitation patterns and does not serve as an accurate approximation to the backward model as we need in the k-space domain design.